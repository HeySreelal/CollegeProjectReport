\documentclass[12pt,a4paper]{report}

\usepackage[left=3cm, right=3cm, top=3cm, bottom=3cm]{geometry}
\usepackage{graphicx}
\usepackage{listings}
\usepackage{titlesec}
\usepackage{fancyhdr}
\usepackage{epstopdf}
\usepackage{float}
\usepackage{amsmath}
\usepackage{setspace}
\usepackage{eufrak}
\usepackage{url}
\usepackage{courier}
\usepackage[nottoc,notlot,notlof]{tocbibind}
\usepackage{acronym}
\usepackage{hyperref}
\newcommand{\textform}[1]{\fontsize{14}{20}\selectfont{#1}}
\pagestyle{fancy}
\fancyhf{}
\rfoot{\thepage}
\renewcommand{\chaptermark}[1]{\markboth{#1}{}}
\renewcommand{\chaptername}{CHAPTER}
\renewcommand{\headrulewidth}{1pt}
\renewcommand{\footrulewidth}{1pt}
\begin{document}
\bibliographystyle{plain}

\thispagestyle{empty}

\begin{center}
    \fontsize{25pt}{18pt}\selectfont \textbf{
        Ticket Booking System for Edakkal Caves
    }\\[.5 cm]
    \fontsize{12pt}{18pt}\selectfont \text{PROJECT REPORT}\\[.5 cm]
    \fontsize{12pt}{18pt}\selectfont \text{Submitted to the}\\[.7 cm]
    \fontsize{14pt}{18pt}\selectfont \textbf{Kannur
        University}\\[.2 cm]
    \fontsize{12pt}{18pt}\selectfont \text{in partial fulfillment of the requirements for the award of }\\[.5 cm]
    \vspace{0.4cm}
    \fontsize{14pt}{18pt}\selectfont \textbf{Bachelor of Science}\\
    \fontsize{12pt}{18pt}\selectfont \text{in}\\
    \fontsize{12pt}{18pt}\selectfont \textbf{\textit{Computer Science}}\\
    \fontsize{12pt}{18pt}\selectfont \text{by}\\
    \vspace{0.4cm}
    \fontsize{17.28pt}{18pt}\selectfont \textbf{SREELAL TS}\\[.2 cm]
    \fontsize{14pt}{18pt}\selectfont \textbf{(MM21CCSR26)}\\[.5 cm]
    \fontsize{12pt}{18pt}\selectfont \textit{under the guidance of}\\[.4 cm]
    \fontsize{14pt}{18pt}\selectfont \textbf{Mr. SABU O J}\\
    \vspace{.7cm}
    \includegraphics[scale=0.25]{assets/mmc.png}\\[.2 cm]
    \fontsize{14pt}{18pt}\selectfont \textbf{PG and Research Department of Computer Science}\\[.5 cm]
    \fontsize{12pt}{18pt}\selectfont \text{Mary Matha Arts and Science College}\\[.4 cm]
    \fontsize{12pt}{14pt}\selectfont \text{Mananthavady}\\[.5 cm]
    \fontsize{14pt}{18pt}\selectfont \text{MAY 2024}\\[.5 cm]
\end{center}



% Declaration Page --------------------------------------------------------------
\newpage
\fontsize{12pt}{20}\selectfont
\renewcommand\abstractname{\textform{\textbf{DECLARATION}}}
\begin{abstract}
    \vspace{1.5 cm}
    \indent
    I hereby declare that the work presented in this project report entitled \textbf{\textquotedblleft Ticket Booking System for Edakkal Caves\textquotedblright}, is based on the original project work carried out by me under the supervision of  \textbf{Mr. Sabu O J}, Assistant Professor, PG and Research Department of Computer Science, Mary Matha Arts \& Science College, Mananthavady affiliated to Kannur University, Kerala.  The project work presented in this report or parts of it has not been presented for the award of any other degree(s). \\[4cm]
    \begin{minipage}{.8\textwidth}
        \begin{flushleft}
            Place : \textbf{Mananthavady}\\
            Date  :
        \end{flushleft}
    \end{minipage}
    \begin{minipage}{.8\textwidth}
        \vspace{1 cm}
        \begin{flushright}
            \begin{flushleft}
                \textbf{Sreelal TS}
            \end{flushleft}
        \end{flushright}
    \end{minipage}
\end{abstract}

% Certificate Page -----------------------------------------------------
\newpage
\fontsize{12pt}{20}\selectfont
\thispagestyle{empty}
\renewcommand\abstractname{\textform{\textbf{CERTIFICATE}}}
\begin{abstract}
    \vspace{1.5 cm}
    \indent
    This is to certify that this project report entitled  \textbf{\textquotedblleft Ticket Booking System for Edakkal Caves\textquotedblright}, is a bonafide record of the work carried out by \textbf{Mr. Sreelal TS} under our supervision in the PG and Research Department of Computer Science, Mary Matha Arts \& Science College, as a part of his/her Bachelor of Science in Computer Science.  The work presented in this project or parts of it has not been presented for the award of any other degree(s). \\[1.5cm]

    \begin{minipage}{.7\textwidth}
        \begin{flushleft}
            GUIDE \\
        \end{flushleft}
    \end{minipage}
    \begin{minipage}{.9\textwidth}

        \begin{flushleft}
            HEAD OF THE DEPT. \\[1 cm]
        \end{flushleft}
    \end{minipage}
    \vfill
    \begin{minipage}{.7\textwidth}
        \begin{flushleft}
            Place:\\
            \vspace{0.2 cm}
            Date:\\
            \vspace{0.8 cm}

            Viva voce held on: {\rule{6cm}{0.5pt}}\\
            \vspace{3cm}
            1) Examiner 1:\\
            \vspace{3cm}
            2) Examiner 2:\\
        \end{flushleft}
    \end{minipage}
\end{abstract}

% Aknowledgement Page -----------------------------------------------------

\begin{spacing}{1.2}
    \tableofcontents
\end{spacing}

\thispagestyle{empty}
\fontsize{12pt}{20}\selectfont
\renewcommand\abstractname{\textform{\textbf{ACKNOWLEDGEMENT}}}
\addcontentsline{toc}{chapter}{ACKNOWLEDGEMENT}
\pagenumbering{roman}
\setcounter{page}{1}
\begin{abstract}
    \vspace{1cm}
    The successful completion of this project would not have been possible without the
    constant support and guidance of many individuals. First and foremost, I thank Software
    Developers across the globe building amazing open source projects such as Flutter, React, etc.
    that became the foundation of this project. I am highly indebted to my institution, Mary Matha Arts
    and Science College, Mananthavady for providing me with the necessary facilities to work on this project.
    I would like to extend my sincere gratitude to
    Dr. Maria Martin Joseph, the principal, and Ms. Jisha T E, Head of Department,
    PG and Research Department of Computer Science, for their constant support.
    I would also like to thank my guide Mr. Sabu O J, for the valuable guidance
    throughout this project work. I also thank the Management and the staff of Mary
    Matha Arts and Science College, Mananthavady for providing me with an opportunity
    to do the project work. Last, but perhaps most important, I thank my parents,
    family members, and friends for their love and continuous support without which
    this work would never have been done.
    \\[2cm]

    \begin{flushleft}
        \hfill
        \fontsize{12}{20}\selectfont {\textbf{SREELAL TS}}
    \end{flushleft}
\end{abstract}
\thispagestyle{empty}
\clearpage


% Abstract Page -----------------------------------------------------
\newpage
\fontsize{12pt}{20}\selectfont
\thispagestyle{empty}
\renewcommand\abstractname{\textform{\textbf{ABSTRACT}}}
\addcontentsline{toc}{chapter}{ABSTRACT}
\pagenumbering{roman}
\setcounter{page}{2}
\begin{abstract}
    \vspace{1cm}
    The Edakkal Caves Ticket Booking System represents a collaborative initiative between the District Tourism Promotion Council (DTPC) and the Incubation \& Innovation Cell at Mary Matha Arts \& Science College, Mananthavady. This project, driven by the vision to enhance the visitor experience to the historical Edakkal Caves in Wayanad, Kerala, introduces an efficient and user-friendly ticket booking system. The system comprises two integral components: the Admin App and the User Portal. The Admin App, built with Flutter and Firebase, empowers DTPC agents with on-site booking management and verification capabilities. Simultaneously, the User Portal, developed using ReactJS and Firebase with Razorpay integration, provides tourists with a seamless online ticket booking experience.
    Visitors can effortlessly navigate the web application, explore details about Edakkal Caves, and reserve tickets for a specified date and time. The integration of Razorpay ensures secure and hassle-free transactions. The historical significance of Edakkal Caves, coupled with the advanced technology employed in this project, aims to make the site more accessible and enrich the overall tourism experience.
    This collaborative effort between the tourism sector and educational institutions exemplifies the potential of leveraging technology to preserve and promote cultural heritage, creating a model for future projects at the intersection of tourism and technology.
\end{abstract}

\newpage
\pagenumbering{roman}
\setcounter{page}{3}
\renewcommand{\baselinestretch}{1.50}
\thispagestyle{empty}
\listoftables
\addcontentsline{toc}{chapter}{LIST OF TABLES}
\fontsize{12pt}{14} \selectfont
\thispagestyle{empty}
\newpage
\pagenumbering{roman}
\setcounter{page}{4}
\renewcommand{\baselinestretch}{1.50}
\thispagestyle{empty}
\listoffigures
\addcontentsline{toc}{chapter}{LIST OF FIGURES}
\fontsize{12pt}{13} \selectfont

\chapter{INTRODUCTION}
\pagenumbering{arabic}
\setcounter{page}{1}
\renewcommand{\baselinestretch}{1.50}
\fontsize{12pt}{14}\selectfont
\thispagestyle{fancy}
\section{Project Overview}

We all know and care about the Edakkal Caves—a testament to the rich historical tapestry of South India. These natural caves bear witness to the Stone Age, adorned with rare carvings that stand as unique relics of our ancient heritage. Recognizing the significance of this cultural treasure, the District Tourism Promotion Council embarked on a mission to enhance the accessibility of Edakkal Caves for enthusiasts and explorers alike.

Our endeavor, "The Ticket Booking System for Edakkal Caves," is more than a technological innovation; it is a gateway to unlocking the wonders concealed within the caverns. This project aims to streamline the ticket booking process, offering a seamless and efficient online solution that transcends geographical constraints. By introducing an intuitive user-facing web application, we empower visitors to effortlessly plan their visits, explore available time slots, and secure their tickets from the comfort of their devices.
This initiative addresses the modern traveler's need for convenience while respecting the historical significance of the Edakkal Caves. The digital ticketing system not only simplifies the reservation process but also contributes to the preservation of this cultural heritage site by minimizing queues and foot traffic.
In addition to catering to the needs of the tourists, our project incorporates a robust administrative component. The development of an Android and iOS admin panel, crafted with Flutter, enables the District Tourism Promotion Council agents to efficiently manage and verify bookings. Through the integration of a secure QR code scanning system, on-site validation becomes a seamless and expedited process, enhancing the overall operational efficiency of managing Edakkal Cave visits.

In this report, we delve into the intricacies of our two-fold solution, detailing the user-facing web application and the administrative mobile app. Through this comprehensive documentation, we aim to showcase not just the technological prowess but also the real-world impact of our project on enhancing the accessibility and preservation of Edakkal Caves.

\section{Problem Statement}

the lack of a streamlined ticketing system posed a significant challenge. Traditional methods led to long queues, hindering the visitor experience and potentially impacting the preservation of this cultural treasure. Recognizing this issue, the District Tourism Promotion Council sought a solution to modernize the ticketing process, making it accessible, efficient, and respectful of the historical significance of Edakkal Caves.

\section{Significance of the Project}

The Ticket Booking System for Edakkal Caves holds paramount importance in bridging the gap between historical preservation and modern convenience. By introducing an online ticketing solution, the project not only enhances the accessibility of Edakkal Caves for tourists but also contributes to the conservation of this cultural heritage site.

\section{Existing System \& Its Limitations}

Presently, ticketing at Edakkal Caves relies on a manual counter-based approach, where visitors procure tickets on-site. While this traditional method has served its purpose, it comes with inherent limitations. Long queues and potential delays at the ticket counter detract from the overall visitor experience, leading to frustration and inefficiencies. Moreover, the manual system poses challenges in efficiently managing visitor data and validating tickets, impacting the administrative processes of the District Tourism Promotion Council. The need for a modernized approach is evident, calling for a transition to an online ticketing system that not only addresses these limitations but also aligns with contemporary expectations of accessibility and efficiency.

\section{Future Scopes}

Beyond its current ticketing functionality, the Ticket Booking System for Edakkal Caves holds immense potential for evolving into a dynamic information platform. As technology advances, envisioning the system as a comprehensive information panel about Edakkal Caves is a logical progression. By leveraging this platform, District Tourism Promotion Council agents can seamlessly disseminate the latest updates, historical insights, and relevant information to visitors. This expansion aligns with the broader goal of transforming the project into a multifaceted tool that not only simplifies ticketing logistics but also serves as an immersive resource for enhancing visitor engagement and knowledge.


\end{document}